\documentclass[a4paper,11pt]{article}
\usepackage[utf8]{inputenc}
\usepackage[T1]{fontenc}
\usepackage[english]{babel}
\usepackage{amsmath,amssymb,amsthm,amsopn}
\usepackage{mathrsfs}
\usepackage{graphicx}
%\usepackage{tikz}
%\usepackage{array}
%\usepackage[top=1cm,bottom=1cm]{geometry}
%\usepackage{listings}
%\usepackage{xcolor}
\usepackage{hyperref}
\hypersetup{
    colorlinks=true,
    linkcolor=blue,
    citecolor=red,
}


% fancy headers and footers

%\usepackage{fancyhdr}
%\pagestyle{fancy}
%\fancyhead[L]{BCPST 2 - Lycée Jacques Prévert}
%\fancyhead[R]{Rappels d'analyse}
%\pagenumbering{gobble} % no page numbering

% Création des labels Théorème, Lemme, etc...

\newtheoremstyle{break}%
{}{}%
{\itshape}{}%
{\bfseries}{}%  % Note that final punctuation is omitted.
{\newline}{}

\newtheoremstyle{sc}%
{}{}%
{}{}%
{\scshape}{}%  % Note that final punctuation is omitted.
{\newline}{}

\theoremstyle{break}
\newtheorem{thm}{Theorem}[section]
\newtheorem{lm}[thm]{Lemma}
\newtheorem{prop}[thm]{Proposition}
\newtheorem{cor}[thm]{Corollary}

\theoremstyle{sc}
\newtheorem{exo}{Exercise}

\theoremstyle{definition}
\newtheorem{defi}[thm]{Definition}
\newtheorem{ex}[thm]{Example}

\theoremstyle{remark}
\newtheorem{rem}[thm]{Remark}

% Raccourcis pour les opérateurs mathématiques (les espaces avant-après sont modifiés pour mieux rentrer dans les codes mathématiques usuels)
\DeclareMathOperator{\Ker}{Ker}
\DeclareMathOperator{\Id}{Id}
\DeclareMathOperator{\Img}{Im}
\DeclareMathOperator{\Card}{Card}
\DeclareMathOperator{\Vect}{Vect}
\DeclareMathOperator{\Tr}{Tr}


% Nouvelles commandes
\newcommand{\ps}[2]{\left\langle#1,#2\right\rangle}
\newcommand{\ent}[2]{[\![#1,#2]\!]}
\newcommand{\diff}{\mathop{}\!\mathrm{d}}
\newcommand{\ie}{\emph{i.e. }}

% opening
\title{}
\author{Understanding the powers-of-2 descent}


\begin{document}

\maketitle

%\begin{abstract}

%\end{abstract}

%\tableofcontents

%\clearpage

\section{Introduction}

The discrete logarithm problem has seen dramatic improvements these last
years in the small characteristic case~\cite{Joux13, BGJT13, GKZ14},
culminating with two \emph{quasi-polynomial} algorithms in 2014. ``On the powers
of $2$'' is the paper~\cite{GKZ14} introducing one of those two algorithms and this document is
meant to give some explanations about it. Before going into details about this
algorithm, we recall a few facts about discrete logarithm. A more complete
panorama can be found in suveys such as~\cite{JP16, GKZ16}.

\section{The Discrete Logarithm Problem (DLP)}
Let $G=\left\langle g\right\rangle$ be a cyclic group generated by an element
$g$, and denote by $N=|G|$ its cardinal. We have the isomorphism:
\[
 \begin{array}{cccc}
   exp_g: & \mathbb{Z}/N\mathbb{Z} & \to & G \\
   & n & \mapsto & g^n,
 \end{array}
\]
and we denote by $\log_g=\exp_g^{-1}$ the inverse isomorphism. In practice, the
\emph{square and multiply} algorithm allows us to compute $g^n=\exp_g(n)$
efficiently, \ie in polynomial time. But, given $y = g^k$, the computation of $k
= \log_g(y)$ is not as easy. This kind of function $f$, where $f$ is easy to
compute but $f^{-1}$ is hard to compute, are called \emph{one-way} functions. They are
typically used in cryptology to make the encryption fast and the deciphering
slow. The DLP first appeared in the article of Diffie and
Hellman in 1976, ``New Directions in Cryptography''~\cite{DH76}. It was supposed to be a very hard problem,
\ie only an exponential time algorithm was known at that time, and all the
security of the protocol invented in this article relied on the hardness of the
DLP.
\section{The index calculus method}



\clearpage
\bibliographystyle{plain}
\bibliography{dlog}
\end{document}
