\documentclass[a4paper,11pt]{article}
\usepackage[utf8]{inputenc}
\usepackage[T1]{fontenc}
\usepackage[french]{babel}
\usepackage{amsmath,amssymb,amsthm,amsopn}
\usepackage{mathrsfs}
\usepackage{graphicx}
%\usepackage{tikz}
%\usepackage{array}
%\usepackage[top=1cm,bottom=1cm]{geometry}
%\usepackage{listings}
\usepackage{xcolor}
\usepackage{hyperref}
\hypersetup{
    colorlinks=true,
    linkcolor=blue,
    citecolor=red,
}

\newtheoremstyle{break}%
{}{}%
{\itshape}{}%
{\bfseries}{}%  % Note that final punctuation is omitted.
{\newline}{}

\newtheoremstyle{sc}%
{}{}%
{}{}%
{\scshape}{}%  % Note that final punctuation is omitted.
{\newline}{}

\theoremstyle{break}
\newtheorem{thm}{Theorem}[section]
\newtheorem{lm}[thm]{Lemma}
\newtheorem{prop}[thm]{Proposition}
\newtheorem{cor}[thm]{Corollary}

\theoremstyle{sc}
\newtheorem{exo}{Exercise}

\theoremstyle{definition}
\newtheorem{defi}[thm]{Definition}
\newtheorem{ex}[thm]{Example}

\theoremstyle{remark}
\newtheorem{rem}[thm]{Remark}

\DeclareMathOperator{\Ker}{Ker}
\DeclareMathOperator{\Id}{Id}
\DeclareMathOperator{\Img}{Im}
\DeclareMathOperator{\Card}{Card}
\DeclareMathOperator{\Vect}{Vect}
\DeclareMathOperator{\Tr}{Tr}


% Nouvelles commandes
\newcommand{\ps}[2]{\left\langle#1,#2\right\rangle}
\newcommand{\ent}[2]{[\![#1,#2]\!]}
\newcommand{\diff}{\mathop{}\!\mathrm{d}}
\newcommand{\ie}{\emph{i.e. }}
% know what is going on or when I still need to ckeck

% opening
\title{État des lieux du stage}
\author{Édouard \textsc{Rousseau}}


\begin{document}

\maketitle

%\begin{abstract}

%\end{abstract}

%\tableofcontents

%\clearpage

\section{Encadrement}

J'effectue mon stage au Canada, à l'université de Waterloo, sous la direction
d'Éric \textsc{Schost}. Nous discutons également très régulièrement avec Luca
\textsc{De Feo}. Je suis arrivé au Canada le 16 Mars et devrait repartir le 8
Septembre.

\section{Sujet}

Le stage porte sur l'étude du problème du logarithme discret dans les corps
finis de petite caractéristique. Le problème est relativement simple à
expliquer : soit $G=\left\langle g\right\rangle$ un groupe cyclique engendré par
un élément $g$, de cardinal $N$. On a alors un isomorphisme :
\[
   \begin{array}{cccc}
        exp_g: & \mathbb{Z}/N\mathbb{Z} & \to & \mathcal G \\
        & n & \mapsto & g^n,
   \end{array}
\]
dont on note $\exp_g^{-1}=\log_g$ (ou plus simplement $\log$) l'isomorphisme
inverse. L'essence du problème vient du fait que, en pratique, on dispose
d'algorithmes efficaces pour calculer $g^n=\exp_g(n)$ (les algorithmes de types
\emph{square and multiply}), mais que cela n'est plus vrai pour les calculs de
l'isomorphisme inverse. En effet, étant donné un élément $y=g^n$, on ne dispose
pas d'algorithme aussi efficace pour calculer $n = \log y$.
Historiquement, Gauss s'intéressait déjà à des
calculs de logarithmes discrets, qu'il appelait ``indices'', mais le problème
est véritablement devenu célèbre en 1976 suite à l'article ``New directions in
cryptography'' de Diffie et Hellman~\cite{DH76}. Au cœur de cet article, qui 
est à l'origine même du concept de cryptographie à clé publique, on trouve le 
protocole d'échange de clés Diffie-Hellman. La sécurité de ce protocole 
s'appuie sur la difficulté supposée de calculer des logarithmes discrets dans le
groupe $(\mathbb{Z}/N\mathbb{Z})^\times$. 

\bibliographystyle{plain}
\bibliography{dlog}
\end{document}
