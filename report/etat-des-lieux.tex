\documentclass[a4paper,11pt]{article}
\usepackage[utf8]{inputenc}
\usepackage[T1]{fontenc}
\usepackage[french]{babel}
\usepackage{amsmath,amssymb,amsthm,amsopn}
\usepackage{mathrsfs}
\usepackage{graphicx}
%\usepackage{tikz}
%\usepackage{array}
%\usepackage[top=1cm,bottom=1cm]{geometry}
%\usepackage{listings}
\usepackage{xcolor}
\usepackage{hyperref}
\hypersetup{
    colorlinks=true,
    linkcolor=blue,
    citecolor=red,
}

\newtheoremstyle{break}%
{}{}%
{\itshape}{}%
{\bfseries}{}%  % Note that final punctuation is omitted.
{\newline}{}

\newtheoremstyle{sc}%
{}{}%
{}{}%
{\scshape}{}%  % Note that final punctuation is omitted.
{\newline}{}

\theoremstyle{break}
\newtheorem{thm}{Theorem}[section]
\newtheorem{lm}[thm]{Lemma}
\newtheorem{prop}[thm]{Proposition}
\newtheorem{cor}[thm]{Corollary}

\theoremstyle{sc}
\newtheorem{exo}{Exercise}

\theoremstyle{definition}
\newtheorem{defi}[thm]{Definition}
\newtheorem{ex}[thm]{Example}

\theoremstyle{remark}
\newtheorem{rem}[thm]{Remark}

\DeclareMathOperator{\Ker}{Ker}
\DeclareMathOperator{\Id}{Id}
\DeclareMathOperator{\Img}{Im}
\DeclareMathOperator{\Card}{Card}
\DeclareMathOperator{\Vect}{Vect}
\DeclareMathOperator{\Tr}{Tr}


% Nouvelles commandes
\newcommand{\ps}[2]{\left\langle#1,#2\right\rangle}
\newcommand{\ent}[2]{[\![#1,#2]\!]}
\newcommand{\diff}{\mathop{}\!\mathrm{d}}
\newcommand{\ie}{\emph{i.e. }}
% know what is going on or when I still need to ckeck

% opening
\title{État des lieux du stage}
\author{Édouard \textsc{Rousseau}}


\begin{document}

\maketitle

%\begin{abstract}

%\end{abstract}

%\tableofcontents

%\clearpage

\section{Encadrement}

J'effectue mon stage au Canada, à l'université de Waterloo, sous la direction
d'Éric \textsc{Schost}. Je suis arrivé au Canada le 16 Mars et devrais repartir le 8
Septembre.

\section{Sujet}

\paragraph{Arrière-plan.} 

Le stage porte sur l'étude du problème du logarithme discret dans les corps
finis de petite caractéristique. Le problème est relativement simple à
expliquer : soit $G=\left\langle g\right\rangle$ un groupe cyclique engendré par
un élément $g$, de cardinal $N$. On a alors un isomorphisme :
\[
   \begin{array}{cccc}
        exp_g: & \mathbb{Z}/N\mathbb{Z} & \to & \mathcal G \\
        & n & \mapsto & g^n,
   \end{array}
\]
dont on note $\exp_g^{-1}=\log_g$ (ou plus simplement $\log$) l'isomorphisme
inverse. L'essence du problème vient du fait que, en pratique, on dispose
d'algorithmes efficaces pour calculer $g^n=\exp_g(n)$ (les algorithmes de types
\emph{square and multiply}), mais que cela n'est plus vrai pour les calculs de
l'isomorphisme inverse. En effet, étant donné un élément $y=g^n$, on ne dispose
pas d'algorithme aussi efficace pour calculer $n = \log y$.
Historiquement, Gauss s'intéressait déjà à des
calculs de logarithmes discrets, qu'il appelait ``indices'', mais le problème
est véritablement devenu célèbre en 1976 suite à l'article ``New directions in
cryptography'' de Diffie et Hellman~\cite{DH76}. Au cœur de cet article, qui 
est à l'origine même du concept de cryptographie à clé publique, on trouve le 
protocole d'échange de clés Diffie-Hellman. La sécurité de ce protocole 
s'appuie sur la difficulté supposée de calculer des logarithmes discrets dans le
groupe $(\mathbb{Z}/N\mathbb{Z})^\times$. D'autres groupes ont un intérêt
cryptographique, par exemple le groupe des points d'une courbe
elliptique, ou bien le groupe des éléments inversible d'un corps fini.
C'est d'ailleurs ce dernier cas auquel nous nous intéressons pendant ce stage.
Plus précisément, nous étudions les corps finis de \emph{petite
caractéristique}, c'est-à-dire les corps $\mathbb{F}_{q^n}$ où $q$ est petit
devant la taille du corps ($q\ll q^n$).

\paragraph{Un problème difficile ?}

Bien que des algorithmes \emph{sous-exponentielles} aient été
découverts, le problème du logarithme discret n'a pas réellement connu
d'avancées majeures pendant les trentes dernières années. C'était en tous cas
le cas avant 2013, où Antoine Joux a publié un article~\cite{Joux13} avec des
idées novatrices dans le cas des corps finis de petite
caractéristique. Dès l'année suivante, deux
algorithmes~\cite{BGJT13, GKZ14} \emph{quasi-polynomiaux}, tous deux issus de ces
mêmes idées mais indépendants l'un de
l'autre, ont vu le jour, toujours en petite caractéristique. Par quasi-polynomial, on entend une compléxité
de la forme $l^{O(\log l)}$ où $l$ est la taille de l'entrée (ici $l=\log
(q^n)$). Nous ne sommes donc plus très loin de la complexité
polynomiale (\ie de la forme $l^{O(1)}$), qui ferait définitivement passer 
le problème du logarithme discret dans la catégorie des problèmes
``faciles'', en petite caractéristique du moins. Cependant, si ces
algorithmes réalisent une grande avancée en terme de complexité
assymptotique, leur intérêt pratique est beaucoup plus incertain, et il n'existe
pas encore d'implémentation complète de ces algorithmes.

\section{Objectifs}

Notre but est de comprendre les nouveaux algorithmes
quasi-polynomiaux~\cite{BGJT13, GKZ14}, en particulier d'étudier le côté
pratique de ces algorithmes. Pour aller dans ce sens, nous avons implémenté
l'algorithme~\cite{BGJT13}. Dans la seconde moitié du stage, nous
envisageons d'implémenter le second algorithme~\cite{GKZ14}, qui est
actuellement le plus efficace en pratique, et éventuellement d'en proposer
des améliorations.
\bibliographystyle{plain}
\bibliography{dlog}
\end{document}
